% \documentclass[
% aspectratio=169,
% 14pt,
% professionalfonts
% ]{article}

\documentclass{article}
\usepackage[utf8]{inputenc}

\usepackage[a4paper, margin=2cm]{geometry}
\usepackage{hyperref}

\title{Bayesian inference in particle physics}

\author{\underline{Lorenz G\"artner}$^{1}$, Toni Sculac, Judith Katzy }

\date{\today}

\begin{document}
\maketitle


\section{Bayesian Higgs}
We got our hands on some fresh LHC data. It seems to be invariant mass measurements of the channel $H\to \gamma \gamma$ in the region $m \in [100, 160]~\mathrm{GeV}$.

Excitedly, you tell all your friends. One of them tells you that you should analyse the data to obtain a measurement of the signal strength $\mu$. She also tells you, that for this channel the background is expected to have an exponential mass distribution of
$$p(m|m_b, \lambda_b) =\frac{1}{\lambda_b} \exp\left(\frac{m-m_b}{\lambda_b}\right),$$
where $m_b=100~\mathrm{GeV}$ and  $\lambda_b=20~\mathrm{GeV}$. The signal has an expected mass distribution given by
$$p(m|m_b, \lambda_b) = \mathcal{N}(m_s, \sigma_s),$$
where $m_s=125~\mathrm{GeV}$ and $\sigma_s=2~\mathrm{GeV}$. You expect $10000$ background events and $1000$ signal events

\textit{Hint: Always work with twice negative logarithmic probabilities throughout the exercise.}
\begin{enumerate}
    \item Generate MC events for signal and background and load the data. %provide this

    \item Construct the negative log-likelihood function
    $$p(n|\mu, \mu_{bkg}) = \prod_{\mathrm{bins}~b} \mathrm{Poisson}(n_b|\nu_b(\mu, \mu_{bkg}))$$
    where $n_b$ are your measured data yields per bin $b$. The expected yields per bin are
    $$\nu_b(\mu, \mu_{bkg}) = \mu n_{sig,b} + \mu_{bkg} n_{bkg, b},$$
    where $n_{sig,b}$ and $n_{bkg,b}$ are the expected signal and background yields per bin, respectively. Our model parameters are the normalization factors $\mu$ and $\mu_{bkg}$. The goal of this analysis will be to infer $\mu$.

    \item Think about possible prior choices here. We are relatively confident in the background modeling, but are not so confident in the signal normalization. Think about possible prior choices for $\mu$ and $\mu_{bkg}$.
    % $$p(\mu_{bkg}) = \mathcal{N}(\mu_{bkg}|1, 0.5).$$

    \item Combine (twice negative log) likelihood and (twice negative log) prior into a (twice negative log) posterior.

    \item What is the posterior mode? What is the impact of your priors here?

    % \item \textit{Optional:} Perform a fit to your simulated data and obtain first estimates for $\mu$, $\mu_{bkg}$ and their uncertainties. You can fit both the unconstrained likelihood and the constrained likelihood and investigate differences.

    \item {Advanced}: Code up your own implementation of the Metropolis-Hastings algorithm.

    \textit{Hint}: Implement it such that it works with twice negative logarithmic probabilities, for numerical stability.

    \textit{Alternative}: Copy my implementation from \href{https://github.com/lorenzennio/icsc-bayesian-inference/blob/main/mcmc.py}{here}.

    \item Generate samples from the posterior.

    \textit{Bonus}: Make trace plots.

    \item Use the samples to obtain estimates for the 1-and 2-dimensional marginal posteriors of $\mu$ and $\mu_{bkg}$.

    \textit{Hint}: \href{https://corner.readthedocs.io/en/latest/}{Corner plots} are useful here.

    \item What is the 95\% upper credible interval on $\mu$?

    \item What are the 68\% and 95\% highest density intervals of $\mu$ and $\mu_{bkg}$?
    

\end{enumerate}

\end{document}

expon.pdf(y) / scale`` with
``y = (x - loc) / scale``